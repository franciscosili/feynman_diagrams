
\documentclass{article}
\usepackage[T1]{fontenc}
\usepackage{graphicx}
\usepackage{underscore}
\usepackage{amsmath}
\usepackage{feynmp-auto}
\usepackage[landscape]{geometry}

\begin{document}
\pagestyle{empty}\large
\begin{fmffile}{feyngraph}
    \begin{fmfgraph*}(170,90)

        \fmfleft{i2,i1}
        \fmfright{o2,o1}

        \fmfblob{20}{t1}

        \fmflabel{\(g\)}{i1}
        \fmflabel{\(q\)}{i2}
        
        \fmflabel{\(\gamma\)}{o1}
        \fmflabel{\(g\)}{o2}

        \fmf{gluon}{i1,v1}
        \fmf{fermion}{i2,v2}
        
        \fmf{fermion,label=\(q\)}{v2,v1}
        
        \fmf{phantom,tension=0.5}{t1,o1}
        \fmf{phantom,tension=2.0}{t1,v1}
        % \fmf{phantom,tension=0.1}{i1,i2}

        
        
        \fmf{plain}{v1,t1}
        \fmf{photon}{t1,o1}
        \fmf{gluon}{v2,o2}
        
        % instructions https://wiki.physik.uzh.ch/cms/latex:feynman
    \end{fmfgraph*}
\end{fmffile}
\end{document}
